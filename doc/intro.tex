\newcommand{\StorageTiers}%
  {
  \BI
  \I A local, high performance, non-persistent ramdisk.  This
     is where images are initially captured.
  \I A local persistent disk (solid state SSD or spinning HDD).
     This is where interactive viewing or video downloads come from.
  \I A remote archive.  This ensures image retention even if
     the local storage server is lost or damaged.
  \EI
  }

\section{Introduction}

  \PRODUCT{} is an open source system for capturing video from one or
  more network-attached video cameras and for displaying, storing locally
  and archiving remotely captured images.

  \PRODUCT{} can run on inexpensive PC hardware and draw images from
  inexpensive cameras.  It was developed on Ubuntu but should be easily
  portable to any Linux distribution and with minor effort any Unix-based
  system.

  \PRODUCT{} is suitable for capturing, storing and viewing security
  video footage in a small setting such as a house or apartment up
  to a medium to large setting such as a multi-floor office building.
  A modestly equipped PC (sub \$1000, no GPU) should be able to capture
  and process image data from at least 20 or 30 cameras concurrently.

  Key features of \PRODUCT{} include:
  \BE
  \I Inspecting each captured image to determine whether it should
     be stored or discarded.
  \I Only storing images if either motion is detected or a threshold
     time interval has elapsed since the last stored image.
  \I Displaying stored images via a web UI -- all user interaction,
     once the system is installed, is via a web browser.
  \I Migrating image files across three storage tiers:
     \StorageTiers{}
  \I Dealing gracefully with inexpensive, unreliable and poorly connected
     cameras, for example by periodically attempting to reconnect.
  \I Support for multi-tenancy, in both the physical building sense and
     logical access sense of the word.  One process can capture video
     data from multiple cameras in a building.  Different URLs are used
     to present video to different tenants -- each from a distinct subset
     of cameras.
  \I Interactive and off-line viewing.  Interactive via a web UI, which
     allows a viewer to navigate through captured images (i.e., those where
     motion was detected or time had elapsed) or to download a video file
     or image archive consisting of the same files.
  \EE
