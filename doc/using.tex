\clearpage
\section{Using \PRODUCT{}}

  Once \PRODUCT{} has been installed and configured, it works as follows:

  \BE
  \I A process -- \texttt{cam-monitor} runs in the background.
  \I \texttt{cam-monitor} launches one command per camera to generate
     image files.  The usual command is \texttt{ffmpeg}.
  \I The image generator produces image files on a RAMDISK.  This is
     normally mounted in \texttt{/mnt/ramdisk}.
  \I The \texttt{cam-monitor} process inspects images as they are
     generated and when it either detects motion or a timeout
     has passed, it stores images in a backup location, such as
     \texttt{/data/security}.
  \I Periodically, \texttt{cam-monitor} may also run a command to
     copy recent images from the backup location to a remote
     location, for example to protect against damage to or loss of
     the video surveillance system.
  \I A web server is configured on the video surveillance system.
     This is normally Apache2.
  \I Users can view recently backed up images using a web browser.
     The usual URL for this is \URL{http://localhost/cam-bin/cam-view}.
  \I Users can view the status of the system at
     \URL{http://localhost/cam-bin/cam-status}.
  \I With either \texttt{cam-view} or \texttt{cam-status}, users can 
     click on a single image to zoom in, and click on arrow icons on
     either side of a zoomed-in image to step forward or backwards
     through stored images.  Clicking anywhere else in the zoomed in
     image closes the zoom-in popup view.
  \I Users can download recorded images or video at
     \URL{http://localhost/cam-bin/download-recordings}.
  \I A number of shell scripts are scheduled to run periodically:
     \BI
     \I \texttt{/usr/local/bin/cam-view-clean-download.sh} cleans
        old files from the download folder.
     \I \texttt{/usr/local/bin/cam-view-health-check.sh} checks
        that \texttt{cam-monitor} is running and, if not, starts it.
     \I \texttt{/usr/local/bin/cam-view-archive.sh} cleans up old
        files in the backup folder.  First, \texttt{.jpg} image 
        files older than two days are stored in a \texttt{.tar}
        archive.  Second, files older than 21 days are deleted.
        Finally, any log files older than 10 days are removed.
     \EI
  \I Command-line utilities are also included.  These
     provide usage instructions via the \texttt{-h} argument:
     \BI
     \I \texttt{cam-archive} -- archive old \texttt{.jpg} images into
        \texttt{.tar} archives.
     \I \texttt{cam-config} -- test parsing of configuration files.
     \EI
  \I Static web documents -- including image files and JavaScript
     libraries -- are normally placed in \texttt{/var/www/html/cam-view}
     and accessed at \URL{http://localhost/cam-view/}.
  \I CGI programs and their configuration files are normally placed in
     \texttt{/var/www/cam-view-bin} and accessed at
     \URL{http://localhost/cam-bin/}.
  \I HTML documents for the UI are constructed by the CGI programs from
     template files, also installed in \texttt{/var/www/cam-view-bin}.
  \I The web CGI programs, the \texttt{cam-monitor} process, the
     image capture commands and the scheduled jobs all run in the
     security context of an (unprivileged) user, normally \texttt{camview}.
  \EE

