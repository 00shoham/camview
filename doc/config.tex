\clearpage
\section{Configuration}
\label{config}

  \PRODUCT{} is configured using a text configuration file.
  Conventionally, this is called \texttt{config.ini}.  In the
  CGI directory, this is the mandatory filename.

  The configuration file specifies both global parameters and a list of
  cameras, each with its own parameters.  The file format
  supports comments, file inclusion and simple macro expansion.

  \subsection{Syntax}

    \BI
    \I Items in the configuration file are written as single lines of text,
       in the form \emph{variable}=\emph{value}.  Values can include spaces and
       punctuation marks, and are ended at the end of each line.
    \I If the combination of a variable and value is very long,
       a configuration line can be split over two or more lines by
       ending all but the last line of text with a \BS{} character.
    \I Comments are lines that begin with a \# character.
    \I If a \emph{variable} is not one that the
       system understands, it is stored and can be referred to later
       in the file by the expression \emph{\$variable}.
    \I Secondary configuration files are incorporated into the main one
       via the \texttt{\#include "}\emph{filename}\texttt{"} directive.
    \EI

    The following block of text illustrates this syntax:

    \begin{verbatim}
CONFIG_NAME=My Video Surveillance System
EPHEMERAL=/mnt/ramdisk/security
PERSISTENT=/data/security
BASE_DIR=$EPHEMERAL
BACKUP_DIR=$PERSISTENT
DOWNLOAD_DIR=$PERSISTENT/download
LOG_FILE=$EPHEMERAL/monitor.log
CGI_LOG_FILE=$EPHEMERAL/cgi.log
DEFAULT_MINIMUM_CAPTURE_INTERVAL=600
BACKUP_COMMAND=/bin/tar cf - %FILES%\
  | /usr/bin/ssh backup@my-backup-service.org "cd /data/security; tar xpf -"
DEFAULT_MINIMUM_CAPTURE_INTERVAL=600
DEFAULT_MOTION_FRAMES=3

CAPFILE=test-image-%08d.jpg
# Use the same command for every camera of this type.  Just
# substitute the IP address or hostname, user and password
# in the RTSP URL:
BESDER_CAPTURE=/usr/bin/ffmpeg\
  -y\
  -i rtsp://$IPADDR:554/user=$USER_password=$PASS_channel=1_stream=1.sdp\
  -r 1\
  -f image2\
  -vframes 99999999\
  $CAPFILE

# Here is the first camera:
CAMERA=front-entry
USER=admin
PASS=admPASS
IPADDR=192.168.1.20
COMMAND=$BESDER_CAPTURE

# Second camera:
# uses the same creds as the front camera so no need to update those.
CAMERA=back-entry
IPADDR=192.168.1.21
COMMAND=$BESDER_CAPTURE
    \end{verbatim}

  \subsection{Global parameters}

    Each configuration file must specify the following parameters:

    \BI
    \I \texttt{CONFIG\_NAME} -- The name of the system.  This is displayed
       in the web UI.\\
       Example usage: \texttt{CONFIG\_NAME=My Video Surveillance System}

    \I \texttt{BASE\_DIR} -- The directory where real time image capture
       will take place.  It is recommended that a RAM disk be created
       and mounted in \texttt{/mnt/ramdisk} and the base directory be
       defined as\\
       \texttt{/mnt/ramdisk/security}. \\
       Example usage: \texttt{BASE\_DIR=/mnt/ramdisk/security}

    \I \texttt{BACKUP\_DIR} -- The directory to which images are copied
       if motion is detected or time has elapsed.  The \texttt{/data/security}
       directory is suggested.\\
       Example usage: \texttt{BACKUP\_DIR=/data/security}

    \I \texttt{DOWNLOAD\_DIR} -- A directory where archives or movie files
       are temporarily placed, before being downloaded to an authorized
       browser.  Since these can be quite large and a RAM disk is usually
       small, do not use the value set for \texttt{BASE\_DIR}.  A
       sub-folder of \texttt{BACKUP\_DIR} is recommended.\\
       Example usage: \texttt{DOWNLOAD\_DIR=/data/security/downloads}

    \I \texttt{LOG\_FILE} -- The full path to a filename where debug
       and event log information will be written.  Since this can be quite
       a busy file whose contents need not be retained for a long time,
       a folder in \texttt{BASE\_DIR}, on a RAM disk, is a good idea.\\
       Example usage: \texttt{LOG\_FILE=/mnt/ramdisk/security/monitor.log}

    \I \texttt{CGI\_LOG\_FILE} -- Like \texttt{LOG\_FILE} but used by
       the web UI of the system, rather than the video capture process.
       Keeping this in the RAM disk also makes sense.\\
       Example usage: \texttt{CGI\_LOG\_FILE=/mnt/ramdisk/security/cgi.log}
    \EI

    The following parameters are not mandatory but can also be specified:

    \BI
    \I \texttt{DEFAULT\_MOTION\_FRAMES} -- If a camera does not specify
       a different value, copy this many image files (frames) from
       the \texttt{BASE\_DIR} folder to the \texttt{BACKUP\_DIR}
       folder after detecting motion.\\
       Example usage: \texttt{DEFAULT\_MOTION\_FRAMES=3}

    \I \texttt{DEFAULT\_MINIMUM\_CAPTURE\_INTERVAL} -- Even if no motion is
       detected, an image will be copied from the \texttt{BASE\_DIR} location
       to the \texttt{BACKUP\_DIR} location periodically.  The period is
       specified in seconds using this directive.\\
       Example usage: \texttt{DEFAULT\_MINIMUM\_CAPTURE\_INTERVAL=600}

    \I \texttt{BACKUP\_COMMAND} -- If there is any concern about the
       video monitoring server itself being damaged or stolen, images
       can be copied from the \texttt{BACKUP\_DIR} directory to another
       location on the network, such as a VM in the cloud.  This is
       the command used to copy a set of files from the video server
       to the off-site backup location.

       The \%FILES\% macro is expanded
       into a list of filenames of the form \emph{cameraID/imagefile-with-timestamp.jpg}.

       The backup command is executed with the current working directory
       being \texttt{BACKUP\_DIR}.

       Example usage:\\
       \texttt{BACKUP\_COMMAND=/bin/tar cf - \%FILES\%} \BS{}\\
       \mbox{}~~\texttt{| /usr/bin/ssh \emph{backupUser}@\emph{backupSystem} "cd /data/security; tar xpf -"}

       Note that in the above example, an SSH trust relationship should exist, to
       eliminate password prompts.

    \I \texttt{FILES\_TO\_CACHE} -- It would be inefficient to run
       \texttt{BACKUP\_COMMAND} once per image.  This parameter tells
       \PRODUCT{} how many files (approximately) it should collect before
       copying images to the remote server.\\
       Example usage: \texttt{FILES\_TO\_CACHE=50}
    \EI

  \subsection{Defining cameras}

    Each camera is introduced in the configuration file with at least
    the following two commands:

    \BI
    \I \texttt{CAMERA} -- The name of the camera.  It is recommended that
       camera names consist only of letters, digits, periods and the dash
       (-) character. All subsequent parameters relate to this camera.\\
       Example usage: \texttt{CAMERA=My-first-camera}

    \I \texttt{COMMAND} -- The command that \PRODUCT{} should run to
       fetch image files from this camera.  This is normally the
       \texttt{ffmpeg} command acquiring a video stream from an RTSP
       service on the camera.  For example, this works for many
       "BESDER" branded cameras:\\
       \texttt{COMMAND=/usr/bin/ffmpeg -y}\BS{}\\
       \mbox{}~~\texttt{  -i rtsp://\emph{ip-address}:554/user=\emph{UU}\_password=\emph{PP}\_channel=1\_stream=1.sdp}\BS{}\\
       \mbox{}~~\texttt{  -r 1}\BS{}\\
       \mbox{}~~\texttt{  -f image2}\BS{}\\
       \mbox{}~~\texttt{  -vframes 99999999}\BS{}\\
       \mbox{}~~\texttt{  test-image-\%08d.jpg}

       \NOTE{\PRODUCT{} requires that image files generated by the
             capture process begin with \texttt{text-image-} and end
             in \texttt{.jpg}.}
    \EI

    The parameters that follow a \texttt{CAMERA} line pertain to just
    that camera.  In addition to the mandatory \texttt{COMMAND} line,
    there are optional parameters that can be included:

    \BI
    \I \texttt{MINIMUM\_CAPTURE\_INTERVAL} -- same as
       \texttt{DEFAULT\_MINIMUM\_CAPTURE\_INTERVAL} but just for
       the current camera.
    \I \texttt{MOTION\_FRAMES} -- same as
       \texttt{DEFAULT\_MOTION\_FRAMES} but just for
       the current camera.
    \I \texttt{DEBUG} -- set to \texttt{true} or \texttt{false} --
       increases logging for the capture and image processing
       performed for this camera.  Can generate a lot of data so
       do not run with this for a long time -- the RAM disk can
       fill up.
    \EI

  \subsection{Debugging configuration files}

    Because of multi-line inputs, variable expansion and file
    inclusion, it is easy to make mistakes in the configuration
    file.  A mechanism is provided to help diagnose such problems.
    Just add the following directive at the end of the main
    configuration file:

    \texttt{\#print "\emph{debug-output-filename}"}

    When the \texttt{cam-monitor} command is run using a configuration
    file that ends in this directive, the configuration file, with all
    includes, long lines and macros expanded out, to be printed to the
    file \emph{filename}.  The monitor will then exit rather than trying
    to connect to cameras.  Run the monitor as follows to do this:

    \texttt{cam-monitor -c \emph{main-config-filename}}

    You can then inspect the contents of \emph{debug-output-filename}.
    Just don't forget to remove this line from your configuration file
    before trying to move your setup to production, as this command
    terminates the monitor after printing the diagnostic file.

