\clearpage
\section{Command-line utilities and shell scripts}
\label{cli}

  \PRODUCT{} includes a handful of command-line programs and shell
  scripts which are useful in the operation of the system.  These are
  described below.

  \subsection{cam-archive}

    \PRODUCT{} accumulates large numbers of JPEG image files in the
    backup folder.  Over time, these will fill all available drive
    space.  Moreover, on most filesystems, very large folders have
    poor runtime performance.

    It makes sense to (a) collect large numbers of small image files
    into a few large archive files and (b) delete older archive files.
    These two tasks are the job of cam-archive:

    \texttt{cam-archive} must be supplied with a path to the \PRODUCT{}
    configuration file:

    \texttt{cam-archive -config CONFIGFILE}

    All other arguments are optional.  They include:

    \BI
    \I \texttt{-camera \emph{ID}} -- operate on a single, specified
       camera.  Otherwise, \texttt{cam-archive} operates on all configured
       cameras.
    \I \texttt{-days \emph{N}} -- operate on a data from N days in
       the past.  The default is 0, meaning today.
    \I \texttt{-range \emph{N1} \emph{N2}} -- operate on a data from multiple
       days, starting with N1 days ago and until N2 days ago.  For example
       \texttt{-range 0 5}.
    \I \texttt{-count} -- print the number of image files per camera.
    \I \texttt{-tar} -- create one or more ".tar" archive files of images
       matching the above conditions.  There will be one archive file
       per day.  The file9s) will be placed in camera-specific backup
       folders and named after the date.
    \I \texttt{-remove} -- delete \texttt{.jpg} files that were archived.
    \EI

  \subsection{cam-view-archive.sh}

    A simple shell script that:

    \BI
    \I Archives images that are 2 or more days old, for each configured camera.
    \I Deletes archive files that are 21 or more days old.
    \I Deletes log files that are 10 days or older on the RAM disk.
    \EI

    This shell script is normally scheduled to run periodically as
    the \texttt{camview} user.

  \subsection{cam-config}

    Used to diagnose configuration files and to extract the values of
    configuration variables.

    The sole mandatory argument is \texttt{-c \emph{configfile}} - which
    specifies the configuration file on which the program should operate.

    The two optional arguments, used separately (not together) are:

    \BI
    \I \texttt{-listincludes} -- prints a list of files included by the
       main configuration file.  This is useful in scripts that copy
       configuration files, such as at installation time, as it tells
       the script what other files are required.
    \I \texttt{-printvar} -- prints the value of a single variable,
       including any expansions.  This is useful to locate folders,
       such as the \texttt{BASE\_DIR} or \texttt{BACKUP\_DIR}, for use
       in scripts such as \texttt{cam-view-archive.sh}.
    \EI

    This program will also print errors if a configuration file
    cannot be parsed.  If the configuration file includes the
    \texttt{\#print} directive, it will also cause the fully expanded
    configuration to be printed out to the specified file.

  \subsection{cam-view-clean-download.sh}

    When users download archives or movie files of captured images,
    the downloaded material is stored in the \texttt{downloads} folder,
    under the \texttt{BACKUP\_DIR} folder specified in the configuration
    file.  This shell script is used to find and remove older download
    files -- by default those that are at least 30 minutes old.

    This shell script is normally scheduled to run periodically as
    the \texttt{camview} user.

  \subsection{cam-view-health-check.sh}

    This is used to check that \texttt{cam-monitor} is running and, if
    not, to stop any running ffmpeg jobs and restart it.  This is also
    how \texttt{cam-monitor} is usually started - by a scheduled cron job
    that runs this script every few minutes as the user \texttt{camview}.

  \subsection{cam-view-stop.sh}

    This is used to stop any running \texttt{ffmpeg} and
    \texttt{cam-monitor} jobs.  It is normally only used in the context
    of system diagnostics or \PRODUCT{} software development.
