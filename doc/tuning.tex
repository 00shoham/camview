\newcommand{\Variable}[4]%
  {%
  \begin{tabularx}{\linewidth}{lY}
  \emph{Parameter}
  & #1
  \\
  \emph{Built-in default}
  & #2
  \\
  \emph{Global default}
  & \texttt{#3}
  \\
  \emph{Per-camera default}
  & \texttt{#4}
  \\
  \end{tabularx}
  }

\clearpage
\section{Tuning}
\label{tuning}

  \PRODUCT{} is quite configurable, in particular in relation to the
  motion detection algorithm.  To understand the parameters, it is
  important to first understand the algorithm.

  \subsection{Motion detection overview}

    \PRODUCT{} compares sequential pairs of images, which it reads from
    JPEG files on the filesystem (typically on a RAM disk).  For each
    such pair, it tries to determine whether there has been motion, by
    seeing if the images are sufficiently different.

    The challenge with motion detection is that images come in via a lossy
    file format (JPEG) and have noise from other sources, including RTSP
    network transport over (lossy) UDP, changing light levels, optical
    artifacts and natural small movements, such as moving shadows or
    rustling leaves in outdoor images.  No two images are identical,
    even if they encode two scenes that are essentially the same.

    The motion detection algorithm is designed to flag image pairs
    where there has been substantial change, while minimizing "false
    negatives" (failed to detect actual motion) and "false positives"
    (flagged motion where there was just noise).

    The motion detection algorithm also has to be quite fast.  Consider
    a deployment with just 10 cameras, each of which produces a
    "full HD" color image -- i.e., 1920 pixels wide, 1080 pixels
    tall, 3 bytes per pixel.  This means that comparing just two
    images requires processing 1920 x 1080 x 3 x 2 = 12,441,600
    bytes of data.  Comparing a pair of images across each of 10 cameras
    means processing 120MB of data.  If images are acquired at
    just 1 image/second, this is 120MB/second (7.2GB/minute).
    This is all in-memory, as the raw image files are JPEG, typically
    about 10x smaller than the raw bitmaps.

  \subsection{Motion detection algorithm}

    Motion detection is a multi-step process, which proceeds as follows:

    \BE
    \I \textbf{Increase brightness of dark images:}

       First, check if either image is too dark.  If so, increase the
       luminosity of both images by the same amount.  Luminosity is
       increased by calculating the logarithm of each of the red, green
       and blue components of each pixel, multiplying by a factor and
       taking the exponent of the result.

    \I \textbf{Subtract the images to generate a gray-scale:}

       Create a gray-scale image, where each pixel is a measure
       of how far the colors of the corresponding pixels in the
       two images are from one another.  There are details in how
       this comparison is calculated that make it more useful in
       noisy images:

       \BI
       \I Color distance = the sum of the difference between red,
          green and blue colors of the pixels.  All difference
          values are absolute (i.e., no negative values).

       \I If the difference between any of the red, blue or green
          components of two pixels is less than a threshold amount,
          it is assumed to be "near enough" to zero and ignored.

       \I Color distance is compared to each neighboring pixel as
          well.  In other words, each pixel in one image is compared to
          nine pixels in the other image.
          If the images are noisy, a visual element may appear in
          one pixel in one image and a nearby pixel in the other
          image, and such differences can be safely ignored.
          The smallest difference between a pixel in one image
          and its neighbors in the other image is used for the
          gray-scale output.

       \I To minimize computational cost, the 1-pixel edges of the
          input images are assumed to be identical and their differences
          are set to "0."  This is not actually true, but we don't care
          enough to incur the extra computational expense of treating
          edge pixels, which have fewer than 8 neighbors, differently
          than "inside" pixels.
       \EI

    \I \textbf{Remove light or bright "speckles" from the difference:}

       The difference image may still be a bit noisy, with occasional
       bright pixels surrounded by dark pixels or dark pixels surrounded
       by white ones.  Such one-off pixels are just noise and are removed.
       If the average brightness of neighboring pixels is above a
       light threshold and the pixel is below a dark threshold, then
       the pixel brightness is made the same as the average brightness
       of its neighbors.  Similarly, if the average brightness of a
       pixel is below a dark threshold while the neighbors are above
       a light threshold, it is made the same as the average brightness
       of its neighbors.

    \I \textbf{Reduce the gray-scale to a smaller black and white checkerboard:}

       The gray-scale image is reduced by a factor of N, where each NxN
       gray-scale pixels are mapped to a single pixel which is either
       black or white.  For each of the NxN pixels in the gray-scale,
       a value of 1 or 0 is assigned based on whether the brightness
       of that pixel is above a threshold.  Next, corresponding the
       checkerboard pixel is set to 1 if the total number of NxN pixels
       set to 1 is above a second threshold.  Essentially, a checkerboard
       pixel is white if a sufficient number of the NxN gray-scale pixels
       are above a brightness threshold.

    \I \textbf{Count the white pixels:}

       Finally, the number of white pixels in the resulting checkerboard
       is counted.  If they represent at least a threshold
       percent of the image area, then the image pair is considered
       to have motion.  If they are below the threshold, then the
       pair of images are considered to be static.
    \EE

  \subsection{Motion detection parameters}

    The algorithm described above mentions various parameters.
    Each parameter has a built-in default plus the configuration
    file can specify either global default values or per-camera
    values for each parameter.  The parameters are as follows:

    \BE
    \I \textbf{Increase brightness of dark images:}

       \Variable{The pixel value of 'dark' below which image
                 brightness will be increased.  Values are out of
                 255(red) + 255(green) + 255(blue) = 765.}
                {40}
                {DEFAULT\_COLOR\_DARK}
                {COLOR\_DARK}

       \Variable{How much to boost the brightness of dark pixels.
                 New brightness (per color) = exp( log( old brightness) * factor ).}
                {1.5}
                {DEFAULT\_DARK\_BRIGHTNESS\_BOOST}
                {DARK\_BRIGHTNESS\_BOOST}

    \I \textbf{Subtract the images to generate a gray-scale:}

       \Variable{Minimum meaningful difference between the red, green
                 or blue components of corresponding pixels in two
                 images.}
                {40}
                {DEFAULT\_COLOR\_DIFF\_THRESHOLD}
                {COLOR\_DIFF\_THRESHOLD}

    \I \textbf{Remove light or bright "speckles" from the difference:}

       \Variable{Maximum brightness of pixel in an image,
                 out of 765 (R+G+B), if the pixel is to be considered
                "dark."}
                {30}
                {DEFAULT\_DESPECKLE\_DARK\_THRESHOLD}
                {DESPECKLE\_DARK\_THRESHOLD}

       \Variable{Minimum average brightness of neighboring pixels,
                 out of 765, for those pixels to be considered "bright."}
                {200}
                {DEFAULT\_DESPECKLE\_NONDARK\_MIN}
                {DESPECKLE\_NONDARK\_MIN}

       \Variable{Minimum brightness of pixel in an image,
                 out of 765 (R+G+B), if the pixel is to be considered
                 "light."}
                {140}
                {DEFAULT\_DESPECKLE\_BRIGHT\_THRESHOLD}
                {DESPECKLE\_BRIGHT\_THRESHOLD}

       \Variable{Maximum average brightness of neighboring pixels,
                 out of 765, if they are to be considered "dark."}
                {60}
                {DEFAULT\_DESPECKLE\_NONBRIGHT\_MAX}
                {DESPECKLE\_NONBRIGHT\_MAX}

    \I \textbf{Reduce the gray-scale to a smaller black and white checkerboard:}

       \Variable{Size of a square in the difference gray-scale image that
                 will be mapped to a single pixel in the black-and-white
                 checkerboard.  Measured in pixels (both width and height).}
                {8}
                {DEFAULT\_CHECKERBOARD\_SQUARE\_SIZE}
                {CHECKERBOARD\_SQUARE\_SIZE}

       \Variable{Minimum brightness (out of 255) of a pixel in the
                 gray-scale difference image that constitutes a "light"
                 pixel.}
                {100}
                {DEFAULT\_CHECKERBOARD\_MIN\_WHITE}
                {CHECKERBOARD\_MIN\_WHITE}

       \Variable{Minimum number of "light" pixels (as above) in the
                 NxN square area of the gray-scale difference image,
                 before a pixel in the "checkerboard" black and white
                 image should be set to "on."  Note that this is out
                 of a maximum of CHECKERBOARD\_SQUARE\_SIZE squared.}
                {33}
                {DEFAULT\_CHECKERBOARD\_NUM\_WHITE}
                {CHECKERBOARD\_NUM\_WHITE}

       \Variable{Percentage of the checkerboard pixels that are lit up
                 as "white" using the above method before the
                 two images that formed the original difference
                 gray-scale are considered to be different enough
                 to indicate motion.  For example, 0.02\% in a
                 1920x1080 image which is sub-sampled into 8x8
                 blocks means
                 \mbox{0.0002 $\times$ 1920 $\times$ 1080 / ( 8 $\times$ 8 ) = 6 pixels.}}
                {0.02\%}
                {DEFAULT\_CHECKERBOARD\_PERCENT}
                {CHECKERBOARD\_PERCENT}

    \EE

