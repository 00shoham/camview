\clearpage
\section{Installation}

  There are a number of elements required to run \PRODUCT{}:

  \BE
  \I A user and group in whose security context the software runs.
  \I A video acquisition program which
     connects to each camera, fetches a video stream and writes
     image files on the filesystem.  The installer
     is able to install ffmpeg, which is commonly used.
  \I A RAMDISK where image files are stored, to minimize I/O cost.
  \I A backup folder, where images of interest are stored.
  \I A web server, used to present all user interfaces.  The installer
     is able to install and configure Apache2.
  \I A variety of CGI programs that run under the web server,
     to (a) display the most recently stored image from each camera,
     (b) to rewind to previous images, (c) to check on the status of
     the system and all cameras and (d) to download video files or
     archives that contain individual image files.
  \I A variety of scripts and binary programs for archiving and cleaning
     up image files and log files.
  \EE

  \subsection{Simple installation}
  \label{simple}

    There is a fairly extensive installation script, which can configure
    all of the above items.  Like the rest of \PRODUCT{}, this script was
    developed and tested on an Ubuntu Linux system.  It may work on other
    systems, with minor modifications, but has not been tested outside
    of this platform.

    To configure the system, first create a configuration file as
    described in \FULLREF{config}.  The example provided is a good starting
    point.  The installer will, by default, look for a configuration file
    called \texttt{config.\emph{hostname}} where \emph{hostname} is the
    output of the \texttt{uname -n} command on your system.  To keep things
    simple, use this filename for now.

    You must specify at least one camera in your configuration file.
    Video capture is typically performed using the \texttt{ffmpeg}
    command, which in turn requires an RTSP URL for your camera.

    Some tips for configuring cameras:

    \BE
    \I Most cameras are activated using a smart phone app, so start there,
       to attach your camera to your wired or WiFi network.
    \I If you can reach the camera with an Ethernet cable, do so.
       This usually yields better image quality and more consistent
       results than WiFi, which is vulnerable to range issues and
       interference.
    \I Do set a user ID and password on your camera.  Consider using the
       same credentials on all your cameras, for simplicity.  Note these
       credentials, as they will be incorporated into the RTSP URL
       later.
    \I If the camera uses DHCP for its network address, adjust your
       DHCP server, which might be on the video surveillance server itself
       (\texttt{isc-dhcp-server} works well) or possibly on your WiFi
       router to assign the camera a static IP address. Otherwise, the
       system will stop working whenever the camera gets a new address!
    \I Refer to your camera's documentation or search the web for a
       suitable URL for your brand and model of camera.  This is a good
       resource for some cameras: \URL{https://security.world/rtsp/}.
    \I Formulate your RTSP URL with your camera's IP address (or hostname
       if you've added it to /etc/hosts or a DNS service), port number,
       login ID and password.
    \I Test the RTSP URL with a program such as ffmpeg (command-line)
       or vlc (GUI).  For example:\\
       \texttt{ffmpeg -y -i rtsp://\emph{Your-URL} test.mp4} -- this
       should generate a video file in test.mp4 (quit with Ctrl-C) or
       error messages.
    \I Make sure you have a working RTSP URL before incorporating it
       into the configuration file.
    \EE

    Once you have written your configuration file and placed it in the
    configuration folder, test it:

    \texttt{%
    cd src\\
    make\\
    cd ../config\\
    ../src/cam-config -c config.\emph{servername}}

    If the configuration file is good, this will print nothing.
    If there are problems, you will see an error message.

    Once you have a good camera RTSP URL and a good configuration file,
    install the software:

    \texttt{%
    cd src\\
    make\\
    cd installer\\
    ./install.sh}

    You need to do this as a user with permission to run commands as
    \texttt{root} using the \texttt{sudo} command and you will be prompted
    for your password, as the installer needs to perform various actions
    as \texttt{root}.

    This will:
    \BI
    \I Install any missing packages, such as apache2 or ffmpeg.
    \I Create a user called \texttt{camview}.
    \I Install binaries from the \texttt{src} folder to \texttt{/usr/local/bin}.
    \I Create a backup folder and a download folder.
    \I Copy the configuration file to \texttt{/usr/local/etc/cam-view/}.
    \I Create and mount a RAMDISK in \texttt{/mnt/ramdisk}, where image
       files will be captured.
    \I Configure Apache2 to support CGI programs and create a CGI URL
       for the software at \URL{http://localhost/cam-bin}.  The files
       for this are in \texttt{/var/www/cam-view-bin}.
    \I Install shell scripts in \texttt{/usr/local/bin/cam-*.sh} and
       schedule some to run periodically in the security context of
       the \texttt{camview} user, for example to clean up old images and
       log files.
    \EI

    You can reverse all this and uninstall the software easily:

    \texttt{%
    cd installer\\
    ./uninstall.sh}

  \subsection{Troubleshooting the system}

    After creating a configuration file and running the installer,
    check that everything works:

    \BE
    \I Was the \texttt{camview} user created?\\
       \texttt{tail -f /etc/passwd}
    \I Is the \texttt{cam-monitor} process running?\\
       \texttt{ps -ef | grep cam-monitor | grep -f grep}
    \I Is there one \texttt{ffmpeg} process running per camera?\\
       \texttt{ps -ef | grep ffmpeg | grep -f grep}
    \I Is the \texttt{/mnt/ramdisk} RAMDISK mounted?\\
       \texttt{df -h /mnt/ramdisk; ls -l /mnt/ramdisk}
    \I Is there at least one camera folder under\\
       \texttt{/mnt/ramdisk/security} and are
       image files being written there?\\
       \texttt{find /mnt/ramdisk/security -name '*jpg'}
    \EE

  \subsection{Web view of captured images}

    To check on the status of the system, you should be able
    to open this URL:

    \URL{http://localhost/cam-bin/cam-status}

    To view "live" images (just those that merit being stored
    in the backup folder, as they are written there), use this
    URL:

    \URL{http://localhost/cam-bin/cam-view}

    To download a video file or \texttt{.tar} archive of
    individual stored images, use this URL:

    \URL{http://localhost/cam-bin/download-recordings}

  \subsection{Remote backups}

    If you want to protect captured images against theft or damage of
    the capture server, you can configure a command to be run periodically
    to copy images elsewhere over the network -- for example to a VM in
    the cloud.

    The installer will check for a folder under the \texttt{config} folder
    called \texttt{.ssh.\emph{hostname}} where the latter part is the output
    of \texttt{uname -n}.  If this exists, it will copy files from there
    to \texttt{/home/camview/.ssh/} once the \texttt{camview} login account
    is created.

    A good approach to doing this is to create an SSH private and public
    key-pair for the \texttt{camview} user, create an account on the remote
    system to which you will copy images and configure the target account to
    trust the \texttt{camview} user's public key:

    \BI
    \I Login to the video capture server.
    \I \texttt{sudo bash} -- \emph{enter your password}.
    \I \texttt{su - camview}
    \I \texttt{ssh-keygen} -- \emph{press Enter at the prompts.}
    \I \texttt{mkdir /tmp/camview-ssh}
    \I \texttt{cp .ssh/* /tmp/camview-ssh}
    \I \texttt{exit} -- i.e., log out of the \texttt{camview} account.
    \I \texttt{exit} -- i.e., this time, log out of the \texttt{root} shell.
    \I \texttt{mkdir \emph{packagedir}/config/.ssh.`uname -n`}
    \I \texttt{cp /tmp/camview-ssh/* \emph{packagedir}/config/.ssh.`uname -n`}
    \EI

    You can run uninstall and install again, to verify that the installer
    will now create and populate \texttt{/home/camview/.ssh}.

    Next, you have to create an SSH trust relationship.  Copy the file
    \texttt{/home/camview/.ssh/id\_rsa.pub} to the target account on the
    remote system and append its contents to that account's \texttt{authorized\_keys}
    file:

    \BI
    \I \texttt{scp /tmp/camview-ssh/id\_rsa.pub \emph{backupaccount}@\emph{remotesystem}:/tmp/}
       -- \emph{enter the password when prompted.}
    \I \texttt{ssh \emph{backupaccount}@\emph{remotesystem}}
       -- \emph{enter the password when prompted.}
    \I \texttt{ssh-keygen}
       -- \emph{only do this if you have never done so on this account.}
    \I \texttt{cat /tmp/id\_rsa.pub >> .ssh/authorized\_keys}.
    \EI

    Test that a trusted SSH login works from the video capture server to the backup
    server:

    \BI
    \I \texttt{sudo bash} -- \emph{enter your password}.
    \I \texttt{su - camview}
    \I \texttt{ssh \emph{backupaccount}@\emph{remotesystem}}
    \EI

    You should get a login shell with no password prompt.  If this
    does not work, check the SSH configuration.  For example, on
    the remote system, in \texttt{/etc/ssh/sshd\_config}, there
    should be an entry like \texttt{PubkeyAuthentication yes}.

    Once you have a working mechanism for the local \texttt{camview}
    user to SSH to a remote account and system, you can add remote backup
    to the configuration file:

    \texttt{%
    FILES\_TO\_CACHE=50\\
    BACKUP\_COMMAND=/bin/tar cf - \%FILES\%\BS{}\\
      | /usr/bin/ssh \emph{backupaccount}@\emph{remotesystem} "cd security; tar xpf -"
    }

    In the above, the first parameter tells the system to send images to the
    remote backup system once every 50 captured frames.  The second parameter
    specifies the command to run to send images over.

  \subsection{Activating changes to the configuration file}

    If you change the configuration file, for example by adding remote backup
    as described in the previous section, you need to install the new configuration
    file into the running system.  There are two ways to do this:

    \BE
    \I Uninstall and (re)install the software.  This is easiest but also
       the most disruptive.

       \NOTE{Be sure to deploy the SSH key material in
             the config folder under \texttt{.ssh.\emph{hostname}}
             so that it gets installed in the new \texttt{camview}
             user's home directory.}

    \I Change the configuration in place and stop/restart processes:
       \BI
       \I Copy the new configuration file to
          \texttt{/usr/local/etc/cam-view/config.ini}.
       \I Stop the cam-monitor process:\\
          \texttt{sudo bash}\\
          \texttt{/usr/local/bin/cam-view-stop.sh}
       \I Either wait for the process to restart automatically (there
          is a health checking job that runs every 5 minutes) or do it manually:\\
          \texttt{sudo bash}\\
          \texttt{su - camview}\\
          \texttt{/usr/local/bin/cam-view-health-check.sh}
       \EI
    \EE

  \subsection{Requiring passwords for web access}

    The default, simple installation does not require user passwords
    to access any of its URLs.  To add passwords, you can create a
    password file.  The installer will incorporate it into the
    Apache configuration

    \BI
    \I \texttt{cd} \emph{basedir}
    \I \texttt{cd} \texttt{config}
    \I\label{htpasswd} \texttt{htpasswd -c htpasswd.`uname -n`} \emph{username}
    \I Enter the password you would like to use.
    \EI

    Uninstall and reinstall \PRODUCT{} and check that the
    URLs now require a password.  The password file gets
    installed to \texttt{/etc/apache2/cam-view.htpasswd}
    and you can use the \texttt{htpasswd} command (which is a part
    of Apache2) to add users.

    The web configuration in
    \texttt{/etc/apache2/conf-enabled/cam-view.conf} specifies the
    password file to use.

    To use something more sophisticated, for example
    \texttt{libapache2-mod-auth-mellon} for SAML authentication
    via trust of an existing identity provider, you can modify
    \texttt{/etc/apache2/conf-enabled/cam-view.conf} directly but you
    are probably better off editing the relevant configuration file in
    \texttt{installer/web-server} so that your changes take effect on
    the next installation.

  \subsection{Multi-tenant deployments}

    In some cases, you want to capture video from many cameras but
    permit different users to view or download content from only a subset
    of cameras.  This creates a small conflict:

    \BE
    \I The system should have a single configuration file for the
       \texttt{cam-monitor} process to use to fetch images from all cameras.
    \I Each tenant or zone should have a unique URL, with its own,
       distinct configuration and password files, representing credentials
       used by people in that tenant and cameras in the tenant space.
    \EE

    This means we need both a global configuration and per-tenant
    configurations.  Ideally we can do this without having to write the
    same configuration data in duplicate locations.

    To address this, we first have to understand how the various
    \PRODUCT{} binary programs, including \texttt{cam-monitor} and
    the CGI user interfaces, find their configuration files.  They
    all use the same logic:

    \BE
    \I In the current working directory, in a file called \texttt{config.ini}.
    \I If that fails, in \texttt{/usr/local/etc/camview/config.ini}.
    \EE

    With this information, we are ready to write multi-tenant configurations:

    \BI
    \I Create one configuration file with all parameters for all
       configuration files.  In our example, we'll call this file
       \texttt{config.params}.  This should set \texttt{BACKUP\_DIR},
       \texttt{BASE\_DIR}, etc. but not define any cameras.
    \I Create one configuration file per tenant or zone.  For example, call these
       \texttt{config.zone1}, \texttt{config.zone2}, etc.  Near the top of
       each of each zone configuration file, write \texttt{\#include "config.params"}.
    \I Create a "master" configuration file called \texttt{config.ini} in
       \texttt{/usr/local/etc/camview} which includes each of the zones.
    \I Create one CGI folder for each zone or tenant.
    \I In each CGI folder:
       \BI
       \I Copy that zone's configuration file to a local file called
          \texttt{config.ini}.
       \I Place a copy of the \texttt{config.params} file.
       \EI
       Consider using symbolic or hard links instead of copies to minimize
       the risk of later duplication.
    \EI

    Because a multi-tenant configuration is more complex than the simple
    case described earlier in \FULLREF{simple}, the installer needs a bit
    of help to configure Apache2.  To do this:
    
    \BE
    \I Create your own configuration file for Apache2.  Create a file
       called \texttt{apache2-config.\emph{hostname}} in the
       \texttt{config} directory, where \emph{hostname} is the output
       of the command \texttt{uname -n}.
    \I Place the Apache2 configuration in this file.  See an example below.
    \I Create \texttt{\emph{zone}.htpasswd} files for each tenant or zone
       that will require password protected logins, also in the \texttt{config}
       folder.  This was illustrated in \FULLREF{htpasswd}.
    \EE

    Example multi-tenant Apache configuration file:
    \begin{verbatim}
      ScriptAlias /zone1/ /var/www/zone1/
      <Directory "/var/www/zone1">
        AuthType Basic
        AuthName "Restricted Content"
        AuthUserFile /etc/apache2/zone1.htpasswd
        Require valid-user
        AllowOverride None
        Order deny,allow
        Options +ExecCGI -MultiViews +SymLinksIfOwnerMatch
      </Directory>

      ScriptAlias /zone2/ /var/www/zone2/
      <Directory "/var/www/zone2">
        AuthType Basic
        AuthName "Restricted Content"
        AuthUserFile /etc/apache2/zone2.htpasswd
        Require valid-user
        AllowOverride None
        Order deny,allow
        Options +ExecCGI -MultiViews +SymLinksIfOwnerMatch
      </Directory>
    \end{verbatim}

    Be sure to name the configuration and password files in the
    \texttt{config} folder as follows:

    \BI
    \I Main configuration file that includes all zones: \texttt{config.}\emph{hostname}.
    \I Configuration file for a single zone: \texttt{config.}\emph{zone}.
    \I Password file for a single zone: \emph{zone}\texttt{.htpasswd}.
    \EI

    Name zones using \texttt{ScriptAlias}, \texttt{Directory} and
    \texttt{AuthUserFile} directives in the Apache2 configuration file
    as shown in the example.

    The installer should be able to handle things from there.

  \subsection{Internet access to local video}

    In a typical deployment, \PRODUCT{} is installed on a local network,
    which is not reachable from a public Internet address.  If you want
    to access the \PRODUCT{} web UI at a public URL, and if you have a
    VM running on a cloud platform such as AWS or Azure, you can readily
    use port forwarding to accomplish this.

    First, ensure that the cloud-hosted VM with a public, static IP address
    will accept inbound connections and forward them to ports on an SSH
    client via a tunnel.  To do this:

    \BI
    \I Lets call the video capture server \texttt{video} and
       the cloud VM \texttt{cloud}.

    \I Ensure that the SSH service is running on \texttt{cloud}:\\
       \texttt{ps -ef|grep sshd}\\
       If you don't see an SSHD process, install the package:\\
       \texttt{sudo apt install ssh}

    \I Edit \texttt{/etc/sshd\_config}

    \I Add the following line:\\
       \texttt{GatewayPorts yes}

    \I Restart the SSHD service:\\
       \texttt{sudo /etc/init.d/sshd reload}
    \EI

    Next, create an SSH trust relationship:

    \BE
    \I Create a login account on \texttt{video}.  Lets call it
       \texttt{client}.

    \I Create a login account on \texttt{cloud}.  Lets call it
       \texttt{server}.

    \I Login as \texttt{client} on \texttt{video} and create a key-pair
       using the \texttt{ssh-keygen} command.

    \I Login as \texttt{server} on \texttt{cloud} and create a key-pair
       using the \texttt{ssh-keygen} command.

    \I Copy the public key from \texttt{client@video} to
       \texttt{server@cloud} and append it to the
       \texttt{.ssh/authorized\_keys} file:

       \BI
       \I \emph{client@video:~\$} \texttt{scp .ssh/id\_rsa.pub server@cloud:/tmp}
       \I \emph{server@cloud:~\$} \texttt{cat /tmp/id\_rsa.pub >> .ssh/authorized\_keys}
       \EI

    \I Verify that it is now possible to login from 
       \texttt{client@video} to
       \texttt{server@cloud} without a password:

       \BI
       \I \emph{client@video:~\$} \texttt{ssh server@cloud}
       \EI
    \EE

    Next, write a shell script that runs as \texttt{root} on
    \texttt{video} which establishes a "reverse" SSH tunnel: Lets call
    this \texttt{tunnel.sh} and place it in the \texttt{/root} folder
    on \texttt{video}:

    \begin{verbatim}
#!/bin/sh

while :
do
  echo "Opening ssh tunnel"
  echo '' | ssh -M -R '*:8881:localhost:80' -i /home/client/.ssh/id_rsa server@video "sleep 99999"
  sleep 1
done
    \end{verbatim}

    Finally, write a shell script that is periodically run as root
    by the cron scheduler, which checks to see if a tunnel is running
    and starts it if not:

    \begin{verbatim}
#!/bin/sh

PROCESS=`ps -ef|grep '8881:localhost:80'|grep -v grep`

if [ -z "$PROCESS" ]; then
  /root/tunnel.sh &
fi
    \end{verbatim}

    Put this in \texttt{/root/tunnel-check.sh}.

    Finally, schedule this job to run every few minutes, to ensure
    that even if the tunnel process failed for some reason, it gets
    restarted:

    \BI
    \I Edit the cron table:\\
       \texttt{crontab -e}
    \I Add this line of text:\\
       \texttt{*/10 * * * * /root/tunnel-check.sh}
    \EI

    You should now be able to access the video system at this URL:

    \URL{http://cloud:8881/cam-bin/cam-view}

    In the above, as noted earlier, \texttt{cloud} is the hostname or
    IP address of the cloud VM that terminates the reverse SSH tunnel.

